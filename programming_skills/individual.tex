\documentclass[11pt,a4paper]{article}

\usepackage{color,graphicx,listings,wrapfig,hyperref}
\usepackage[margin=2cm]{geometry}

\graphicspath{{./img/}}

\begin{document}
\title{Individual report from Programming Skills}
\author{Michał Kawalec}
\maketitle

\section{Introduction}
We were required to create a computer simulation of a simple predator-prey system.
My language of choice was C++, as I feel its object oriented nature enables much better management of code complexity and separation of concerns than what is possible with C. This is especially true when working with a group of relatively inexperienced programmers, as even a slight misunderstanding of C can then be extremely harmful to the program.

Even though the Simulation itself is a rather simple piece of software we engineered its structure in such a way to make it easily extensible by others and planned for future (large) extensions to the codebase. For more details, see relevant section of the group report.

\section{My role}

I took a lead role in the project, being a Benevolent Dictator\footnote{For explanation of this commonly used phrase in Free software, see \url{http://en.wikipedia.org/wiki/Benevolent_Dictator_for_Life}} for the project duration. I decided on the tools we are going to use for project management and suggested the libraries we chose to provide some of the functionality.

I suggested \texttt{git} for source management, as it is much more widely supported than the alternatives providing similar functionality (most notably \texttt{Mercurial} and \texttt{Bazaar}). I think that the inherent drawbacks of centralized source management which include a single point of failure, complex privilege management and the requirement to be online when committing changes make it difficult to defend using centralized repositories for modern software projects. 

The lowest version of \texttt{gcc} on our systems was 4.6, so I proposed using the parts of C++11 standard implemented in this version where applicable and supplement the needed missing functionality with Boost libraries. 
The functionality provided by Boost is the test framework, program arguments parser and Mersenne-Twister random number generator. I thought that since we need only basic unit testing functionality, using unit testing framework which is a part of the same package as other libraries we use has a marked advantage. First of all, Boost is widely used and thus easily available on multiple platforms. Secondly it is written in one style and presents a uniform interface throughout its parts, which reduces occurrence of common errors. 
The same reasoning made me push for using an options parser from Boost. It is easy to use, implements the 'do not repeat yourself' principle and Boost source code is full of examples of using it in various scenarios. 

We decided to use Mersenne-Twister as a random number generator as it provides much better statistics than the standard linear congruential bit included in C++ libraries\footnote{For a longer discussion about the merits of MT, see \url{http://en.wikipedia.org/wiki/Mersenne_twister#Advantages}}. MT generator is actually included in gcc 4.8 as it is a part of C++11 spec we are using in code, but as we decided to support gcc 4.6 we use the Boost implementation.

Apart from setting up the build environment and picking libraries we ended up using I also created a project scaffold, ie. the header files and basic data structures found in \texttt{exceptions.cpp} and \texttt{helpers.cpp}. Having this basis set up, we were able to assign the actual implementation of individual functions between us and work in parallel. 

I also came up with the idea to dynamically select a used Serializer class at runtime and the actual implementation of this feature. It is a commonly found pattern in object-oriented languages in which a group of classes act as plug-ins and they register themselves during their instantiation in some object that is guaranteed to exist throughout the lifetime of the program. 

Such an approach has multiple advantages. First of all the computation is completely independent on the serialization method chosen. Secondly, 
\end{document}
